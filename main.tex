%++++++++++++++++++++++++++++++++++++++++
% Don't modify this section unless you know what you're doing!
\documentclass[letterpaper,12pt]{article}
\usepackage{tabularx} % extra features for tabular environment
\usepackage{amsmath}  % improve math presentation
\usepackage{graphicx} % takes care of graphic including machinery
\usepackage[margin=1in,letterpaper]{geometry} % decreases margins
\usepackage{cite} % takes care of citations
\usepackage[final]{hyperref} % adds hyper links inside the generated pdf file
\hypersetup{
	colorlinks=true,       % false: boxed links; true: colored links
	linkcolor=blue,        % color of internal links
	citecolor=blue,        % color of links to bibliography
	filecolor=magenta,     % color of file links
	urlcolor=blue         
}
%++++++++++++++++++++++++++++++++++++++++


\begin{document}

\title{literature review : Fishing net simulations}
\author{A. Sarath krishnan, B. Partner, and C. Partner}
\date{\today}
\maketitle

\begin{abstract}

\end{abstract}


\section{Vol 10 Contributions on theory of fishing gears and related marine systems}
\subsection{Study on hydrodynamic characteristics of net panel using cfd analysis Y.Takahashi}
CFD simulations by treating twines as solid bodies and finding drag force and comparing Cd,Cf from them with experimental works by Dong et al. (2016). The drag and lift coefficients were under estimated comparing the experimental values. They say this is because of the design of shape of the nets. Cd decreases with a decrease in hanging angle when $\theta < 60$. Twine angle : angle between each twine and flow direction.
$$ \phi = \cos^{-1}{(\cos{\phi}\cos{\theta})}$$

Note : Tauti’s (1934) theory on hydrodynamic forces of net panels. (Read)

Drag coefficients decreases with a decrease in the hanging angle at small angle of attack due to the decrease in the twine angle. Drag force acting on trawl net may be reduced by choosing the appropriate hanging angle.
 

%----------------------------------------------------------------------------------------

\section{Vol 9 Contributions on the theory of fishing gears and related marine systems}

\subsection{The Hydrodynamic drag coeffs of flat netting at longitudinal flow - V.Naumov, N.Agievich}
Coefficient of hydrodynamic drag of a flat netting (Net parallel to the flow) depends on Reynolds number calculated by taking Length of the netting as characteristic length and \(omega\) : relation of threads area to projected area, and \(small delta\) : thickness on the thread to length of the netting. there exist three different areas of drag where Cd is proportional, inversely proportional and independent of Re(L)

considering the flow over a flat plate, since the netting is not really a hydro-dynamically smooth surface, roughness has to be considered. Ratio of $\Delta/l$ , where $\Delta$ is average irregularities and l is boundary layer thickness plays important role. As 'l' boundary layer thickness increases downstream, this ratio decreases.  so the front and rear will behave differently in terms of hydrodynamic resistance due to roughness.

Re1 : 2.4*10⁵ 
Re2 : 2*10⁶

when ReL less than Re1, inversely proportional to Cd by -0.16
when ReL between Re1 and Re2 proportional to Cd by a power of 0.14
ReL greater than Re2, independent of Rel 
They have provided the equations to find Cd for different velocities and compared it with some experiments conducted by other authors.
%------------------------------------------------------------------------------------------
\subsection{Impact of net shape and angle of attack on flow through net panel : Experimental study - christian Semlow, Mathias Paschen}
Purpose is to provide a validation data for CFD studies (Aquaculture). Hot wire probes and prandtl's tube were the sensors used to find the flow and pressure in the vicinity of the net.
Definition of porosity of the mesh : 
\begin{equation}
    \phi= 1 - \frac{d}{a}*\frac{1}{u_1*u_2}
\end{equation}
Where $u_1$ is the hanging ratio given by $sin\gamma$, $\gamma$ is the angle of arrangement of rhombic mesh. No information about $u_2$ is in the paper. ($u_2$ will be the hanging ratio in other direction.)
They notice a significant increase in velocity in the netting plane area compared to the undisturbed flow. the rate id velocity in the netting plane area corresponds to the porosity and is in accordance with continuity equation. Orientation of the net and hanging ratio have visible impact on velocity distribution.


%-------------------------------------------------------------------------------------------------
\section{Vol 8 : Contributions on theory of fishing gears and related marine systems}
\subsection{The hydrodynamic drag coeffs of flat netting at cross flow- V Naumov, N. Velicanov}
To find the drag coeff of flat netting when flow is parallel to the net. Old tries to find the analytical formula for drag coeffs are given. Inorder to find the critical Reynolds number, experiments are conducted for low Reynolds numbers. ($Re < 150$) The details regarding experimental set up used is given.
Theory of stochastic functions has been used for processing the experimental data. The formula to find the critical reynolds number's is given. Cd does not depend on reynolds number after Re is higher than higher limit critical value ReK. 
%++++++++++++++++++++++++++++++++++++++++++++++++++++++++++++++++++++++++++++++++++++++++++++++++++++++
\section{Vol. 7 Contributions on theory of fishing gears and related marine systems}
\subsection{Numerical investigation for underwater fluid-structure interaction problem : Ilyes MNASSRI, David LE TOUZÉ, Benoit Vincent, Bertrand ALESANDRINI}
Sea water is often considered as a fluid with uniform velocity and intensity and the effects of small turbulances are neglected. This small inaccuracy in the first step of the net prediction leads to a large error in the form of simulated nets and finally drag forces are under estimated. Objective of this CFD package is to provide a quantitative description of entire flow field in terms of pressure and velocity around the knots of submerged net structures. Goal is to couple the structural model with surrounding fluid flow to predict morison's forces more accurately.
%------------------------------------------------------------------------------------------------

\subsection{Experimental analysis of the hydrodynamic coefficients of the net panels in the flume tank in Hirtshals : Nina MADSEN, Kurt HANSEN, Birger ENERHAUG}
experiments with different netting of different opening to provide hydrodynamic drag coefficients of the netting. Some details regarding the experimental set used are given. Used an underwater motion tracking system to investigate more details. values on Cd and Cl at different angle of attacks are given which can be useful for comparisons.A porosity model proposed by Gjosund et al (2010) was tested against the measured data. The correlation among both is very poor. Model lacks in showing the strong dependence of porosity and Reynolds number.
A pure empirical model was developed and fitted with measured values. With the results they claim to be able to represent the dependency of angle of attack and drag,lift coeffs with basic mathematical functions independent of twine diameter and mesh opening.


%+++++++++++++++++++++++++++++++++++++++++++++++++++++++++++++++++++++++++++
\subsection{Hydrodynamic loads on 2 Dimensional sheets of netting within the range of small angles of attack : Mathias PASCHEN, Karsten BREDDERMANN}
Discussions on experimental and numerical analysis of stiff net grid made of aluminium. This study aims in understanding the behaviour of fluid near to the netting and hydrodynamic loads of net grids. Tests are carried out in wind tunnels and end plates were used to avoid 3 dimensional effects. Front plate and aft plate to get a smooth circulation around front panel and to simulate the clogging effect caused by the fish catch. Numerical model used to compare is a solid model with around 6 million cells. They concluded on the enormous effects experimental set up had on their results. Fluid flow close to the netting can be influenced because of the flexibility and complex behaviour of fishing nets and predicting those are essential for  calculating  Cd and Cl accurately. What they are asking for is a very strong coupling between solid and fluid model.
%---------------------------------------------------------------------------



%---------------------------------------------------------------------------------
\section{Chapter: works of E. Christensen and H. Chen}

\subsection{First Paper : Investigations on porous resistance coeffs for fishing net structures}

In the first paper which says about the investigations on the porous resistance coeffs, They have come up with an analytical solution for porous coeffs from physical parameters of fishing nets. Probably they might have modified the porous media flow solver in order to apply their approach in the simulation.  They have both Cm ( for inertial effect due to the presence of porous skelton) and S (resistance force) which consist of porous coeff matrices D and C. D comes in a linear term while C is in quadratic term which is the major part of the resistance force.
About neglecting D : This assumption was further justified by the physical explanation: Fishing nets are composed of twines with very small diameters, typical in the order of millimeters, the quadratic drag force is the dominant force for such kind of marine structures. Inertia and other forces are secondary.
Morison's equation and volume averaged NS for porous media are compared to find the equations for forces acting on the porous media. As morison's force model does not consider the interactions between nets, 2 coefficients a,b were introduced to predict the forces more accurately.Velocity inside porous media and undisturbed velocity have negligible difference and they are assumed to be same!
The equations involve angle of attacks and velocity directions. To implement this in a trawling nets could lead to edit the solver.

VALIDATION : The first validation case is based on the experimental data presented in Patursson (2007) for a plane net panel in current
flow under various attack angles and incoming velocities. ( A concern in velocity reduction from a porous baffle model if it is modeled as surface.! Will it provide the same velocity ( Even directions ) as the flow in reality? Stream lines will be deflecting near the net. But when we  make it as a porous surface, it wont right?)
validating with experiments by Zhan et al (2016) : Totally three kinds of net panels with different solidity ratios were studied in the experiments. The mesh for all the three nets were square diamond pattern. They calculated parameters for obtaining quadratic drag coeffs. Drag force was over predicted up to 40 percent  at 0.5 m/s and 30ª angle of attack. relative errors of most of the cases are less than 20 percent they say.
 Errors vary with different net cases : this shows a lack of good porous resistance coefficients ? for net 3, all drag forces were under estimated showing problems in coefficients a and b.
They have also done validations on current interactions with circular fish cages for aqua culture and wave
interaction with net panels.
They carried out sensitivity analysis on porous resistance coeffs to know if they are in reasonable bounds when taking account uncertainties of numerical model. The uncertainties of the porous resistance coefficients come from the following: The drag force coefficient of the twines for a fishing net was assigned with a 10 percent uncertainty due to, i.e. misreading of the figure for drag force coefficients, difference between the shape of the real twine and a cylinder,etc. The projected area S1 and S2 for the in-plane and out-of-plane twines were varied with 5 percent, since they were usually calculated based on the mesh distance and the overall dimension of the fishing nets, therefore there exists round-off errors.
\subsection{Second paper : Development of a numerical model for fluid-structure interaction analysis
of flow through and around an aquaculture net cage}
More detailed paper on FSI for aquaculture net cages. A coupled problem was solved by porous media model for fluid and lumped mass structural solver with an efficient data exchange via RAM. The structural code was implemented inside the Pimple loop. A deforming mesh technique was used for representing rotating porous zone. A flow chart given for FSI. fluid model took more time than solid model. 

%+++++++++++++++++++++++++++++++++++++++++++++++++++++++++++++++++++++++++++++++++++++++


\section{Numerical simulation of the effects of fish behaviour on flow dynamics around net cage : M.F.Tang,etc}
$k-\omega$ SST and large deformation non-linear FEM coupled to achieve interaction between flow and net cage with one way coupling. Since it is a net cage , they have modelled the velocity distribution caused the movement of fish too. The coefficients are similar to the one in OpenFoam. Not a lot of information regarding the coupling is given. Meshing was done on the net very fine, taking net as a solid surface. Experiment with one end of the net not fixed was carried out. Details regarding fish stocking and their effects are there.


%++++++++++++++++++++++++++++++++++++++++
% References section will be created automatically 
% with inclusion of "thebibliography" environment
% as it shown below. See text starting with line
% \begin{thebibliography}{99}
% Note: with this approach it is YOUR responsibility to put them in order
% of appearance.



\end{document}
